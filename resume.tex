\documentclass{article}
\usepackage{graphicx}
\usepackage{amssymb}
\usepackage{amsmath}
\usepackage{fancyhdr}
\usepackage{framed}
\usepackage{tipa}
\usepackage{multicol}
\usepackage[usenames,dvipsnames,svgnames,table]{xcolor}
\usepackage{tabularx}
\usepackage{ragged2e}
\usepackage[colorlinks=true]{hyperref}

\usepackage{tgadventor}
\usepackage[T1]{fontenc}
\renewcommand*\familydefault{\sfdefault}

\textwidth = 16.51cm
\textheight = 24.86cm
\topmargin = -1cm
\headheight = 0cm
\headsep = 0cm
\oddsidemargin = 0cm
\hoffset = -.5cm
\marginparpush = 0cm
\widowpenalty = 300
\clubpenalty = 300

\definecolor{cadetblue}{HTML}{4A6491}
\definecolor{light-gray}{gray}{.90}
\definecolor{dark-gray}{gray}{.25}

\hypersetup{urlcolor=cadetblue}

\makeatletter
\newcommand{\globalcolor}[1]{
    \color{#1}\global\let\default@color\current@color
}
\makeatother
\AtBeginDocument{\globalcolor{dark-gray}}

\newcolumntype{A}{>{\raggedleft\let\newline\\\arraybackslash}X}
\newcolumntype{C}{>{\centering\let\newline\\\arraybackslash}X}

\newcolumntype{L}[1]{>{\raggedleft\let\newline\\\arraybackslash}p{#1}}
\newcolumntype{R}[1]{>{\raggedright\let\newline\\\arraybackslash}p{#1}}

\begin{document}
\thispagestyle{empty} 

\center \huge{Kelly Joseph Price}
\normalsize

\hspace{.5cm}

{
    \footnotesize
    \centering
    \begin{tabular}{rl}
        754 The Alameda APT 4207\newline
        &
        \href{mailto:kellyjosephprice@gmail.com}{kellyjospehprice@gmail.com}\newline
        \\
        San Jose, CA
        &
        (925) 234-2914
        \\\\
    \end{tabular}
}

\begin{tabularx}{\textwidth}{CCCCC}
    \footnotesize
    \href{https://github.com/kellyjosephprice}{Github}
    &
    \footnotesize
    \href{http://kellyjosephprice.tumblr.com}{Blog}
    &
    \footnotesize
    \href{http://kellyjosephprice.github.io}{Profile}
    &
    \footnotesize
    \href{https://www.linkedin.com/pub/kelly-price/22/760/a11}{Linked-in}
    &
    \footnotesize
    \href{https://careers.stackoverflow.com/kellyjosephprice}{Stack Overflow}
    \\\\
\end{tabularx}

\begin{tabular}{L{2.5cm} R{14cm}}
    \arrayrulecolor{light-gray}
    
    \hline \\

    \large{\textbf{projects}} \\\\
    \normalsize
    
    \href{http://www.ramblr.co}{\large{\textbf{ramblr}}}
    \normalsize
    \newline
    \href{https://github.com/kellyjospehprice/ramblr}{source}
    
    &
    
    A yelp-like single-page application. It was designed with a more
    minimal aesthetic. Backbone consumes a RESTful Rails API. There is a categorical
    autocomplete and search using custom routes. Uses polymorphic associations. 
    The API is optimized with eager loading, russian doll caching, and counter
    caching. Code is DRY'd up using concerns and scopes.

    \\\\
    
    \href{http://klog.io/cloneddit}{\large{\textbf{cloneddit}}}
    \normalsize
    \newline
    \href{https://github.com/kellyjospehprice/cloneddit}{source}
    
    &
    
    A clone of reddit as an attempt to rewrite as much as possible from
    scratch. Rewrote ActiveRecord and ActiveResource. Handles a similiar route
    DSL. Manages sessions and flashes through WEBrick's cookie interface.

    \\ \hline \\

    \large{\textbf{skills}}
    \normalsize

    &

    Ruby, Ruby on Rails, Javascript, jQuery, Backbone.js, SQL, RSpec, git, 
    subversion, HTML, CSS, perl, Linux

    \\ \hline \\

    \large{\textbf{employment}} \\\\
    \normalsize

    February 2013 - May 2014
    
    &

    \textbf{Software Engineer}
    Barracuda Networks

    \begin{description}
        \item[$\cdot$] maintained and rewrote install scripts for Barracuda's Linux 
            (custom and CentOS) for all hardware appliances
        \item[$\cdot$] created a client/server system for profiling lifespans of
            SSDs
        \item[$\cdot$] handled support cases for hardware and OS level issues
        \item[$\cdot$] used agile development methods
    \end{description}

    \\\\
    
    July 2012 - February 2013
    &
    \textbf{Software Engineering Intern}
    Barracuda Networks

    \begin{description}
        \item[$\cdot$] wrote initramfs scripts to automatically repartition and 
            format additional disk space added to VM's
        \item[$\cdot$] maintained build system to convert hardware appliances to VM's
            (VMWare, Hyper-V, Xen, VirtualBox) for all of Barracuda's VM products
    \end{description}

    \\\\
    
    January 2011 - March 2012
    &
    \textbf{Production Roaster}
    Barefoot Coffee
    
    \\\\ \hline \\

    \large{\textbf{education}} \\\\
    \normalsize

    August 2014 & App Academy \\

    \\

    2010 - 2013 & De Anza Community College \\
\end{tabular}

\end{document}
